\documentclass[hyperref,masterofscience]{cgvpub}
%weitere Optionen zum Erg�?nzen (in eckigen Klammern):
%
% bibnum	numerische Literaturschl�?ssel
% final 	f�?r Abgabe
% lof			Abbildungsverzeichis
% lot			Tabellenverzeichnis
% noproblem	keine Aufgabenstellung
% notoc			kein Inhaltsverzeichnis
% twoside		zweiseitig
\author{Xiaoyu Yin}
\title{Translating Natural Language to SPARQL}
\birthday{13. June 1994}
\placeofbirth{Zhumadian}
\matno{4572954}
\betreuer{Dr. Dagmar Gromann}
\bibfiles{literatur}
\problem{Text der Aufgabenstellung...}
\copyrighterklaerung{Hier soll jeder Autor die von ihm eingeholten
Zustimmungen der Copyright-Besitzer angeben bzw. die in Web Press
Rooms angegebenen generellen Konditionen seiner Text- und
Bild"ubernahmen zitieren.}
\acknowledgments{Die Danksagung...}
\abstracten{abstract text english}
\abstractde{ Zusammenfassung Text Deutsch}
\begin{document}
\chapter{ein kapitel}
\section{eine Grafik}
\begin{figure}[htbp]
	\centering
		\includegraphics{test.png}
	\caption{beschriftung}
	\label{fig:diplominf}
\end{figure}


\subsection{Etwas Mathe}

\[
\sum_{i=1}^{100}x_i
\]
noch mehr text
\subsubsection{Verweise auf Literatur}
So kann ich Literatur aus literatur.bib zitieren: \cite{kochbuch}.

\paragraph{etwas quelltext}


\begin{figure}[htbp]
\begin{lstlisting}[frame=trbl]
//comment
for(int i = 0; i < 100;i++)
{
test(i);
}
\end{lstlisting}
\end{figure}

text

\cite*{}
\end{document}
