\documentclass[11pt]{beamer}
\usetheme{Dresden}
\usepackage[utf8]{inputenc}
\usepackage{pifont}
\usepackage{amsmath}
\usepackage{amsfonts}
\usepackage{amssymb}
\author{Xiaoyu Yin}
\title{Translating Natural Language to SPARQL}
\setbeamercovered{transparent} 
%\setbeamertemplate{footline}[frame number]
%\setbeamertemplate{navigation symbols}{} 
\setbeamertemplate{bibliography entry title}{}
\setbeamertemplate{bibliography entry location}{}
\setbeamertemplate{bibliography entry note}{}
%\logo{} 
\institute{TU Dresden} 
\date{10 December 2018} 
%\subject{} 
\begin{document}

\AtBeginSection[]
{
    \begin{frame}
    \frametitle{Outline}
    \tableofcontents[currentsection]
    \end{frame}
}

\begin{frame}
    \titlepage
\end{frame}

\begin{frame}
    \frametitle{Outline}
    \tableofcontents[hideallsubsections]
\end{frame}

\section{Background}

\subsection{Math Word Problem Generation}
\begin{frame}{Math Word Problem Generation}
    \begin{block}{Math Word Problem}
        a word problem is a mathematical exercise where significant background information on the problem is presented as text rather than in mathematical notation.
    \end{block}
    \begin{example}[Math Word Problem]
    John has 4 apples. Jack has 5 apples. How many apples do they have in total?
    \end{example}
\end{frame}

\begin{frame}{Math Word Problem Generation}
    Existing methods:
    \begin{itemize}
    \item requirements $\Rightarrow$ problems meeting requirements
    \item problem and themes $\Rightarrow$ problem in new theme
    \item ...
    \end{itemize}
    Our method (ideal):
    \begin{itemize}
    \item problem $\Rightarrow$ more problems
    \end{itemize}
\end{frame}

\subsection{Question Generation}
\begin{frame}{Question Generation}
    \begin{example}[Math Word Problem]
    \underline{John has \alert{4} apples.}
    \underline{Jack has \alert{5} apples.} How many apples do they have in total?
    \end{example}
    \begin{columns}
    \column<2->{5cm}
    $\Downarrow$
    ~\\
    John and Jack have 9 apples in total. Jack has 5 apples. How many apples does \textbf{John} have?
    \column<3->{5cm}
    $\Downarrow$
    ~\\
    John and Jack have 9 apples in total. John has 4 apples. How many apples does \textbf{Jack} have?
    \end{columns}
    ~\\
    \begin{block}<4->{Our goal}
    Generate a nearly new problem with \textbf{no extra input}
    \end{block}
\end{frame}

\begin{frame}{Question Generation}
    \begin{block}{General QG}
    Generating different kinds of questions from a given sentence or paragraph. (like who, when, why, etc.)
    \end{block}
    \begin{block}{In this project}
    Only a subset of question types is needed.
    \begin{example}<2->
        John has 4 apples.
        \begin{description}
        \item[\ding{55}] Who has 4 apples?
        \item[\ding{55}] Why does John have 4 apples?
        \item[\ding{51}] How many apples does John have?
        \end{description}
    \end{example}
    \end{block}
\end{frame}

\section{Project Structure}
\begin{frame}
\textbf{Old} math word problem\\
$\Downarrow$\\
\begin{enumerate}
    \item Preprocessing
    \item Question Generating
    \item Postprocessing
\end{enumerate}
$\Downarrow$\\
\textbf{New} math word problem
\end{frame}

\subsection{Preprocessing}
\begin{frame}{Preprocessing}
    Prepare the original math word problem input for the other tasks\\
    \begin{block}{Steps}
    \begin{enumerate}
        \item Text Simplification
        \item Pronouns Removal
        \item Sentence Filtering
        \item ...
    \end{enumerate}
    \end{block}
\end{frame}

\begin{frame}{Text simplification}{Preprocessing}
    Keep sentences in simple form, important for question generation
    \begin{example}
    Input: John has 4 apples and Jack has 5 apples.\\
    Output: John has 4 apples. Jack has 5 apples.
    \end{example}
    \begin{block}<2->{Avoid such question:}
    \ding{55} John and Jack have 9 apples in total. How many apples does John have and How many apples does Jack have?
    \end{block}
    ~\\
    Method adopted from \emph{\cite{Heilman2010}}
\end{frame}

\begin{frame}{Pronouns removal}{Preprocessing}
    \begin{example}
    Input: John has 4 apples. \textbf{He} took 3 apples from Jack. \\
    Output: John has 4 apples. \textbf{John} took 3 apples from Jack.
    \end{example}
    ~\\
    \begin{block}<2->{Disambiguate the generated question}
    \ding{55} How many apples did \textbf{he} take from Jack?\\
    \ding{51} How many apples did \textbf{John} take from Jack?
    \end{block}  
    ~\\~\\
    Acheived by using \emph{Coreference Resolution} 
\end{frame}

\begin{frame}{Sentence Filtering}{Preprocessing}
    \begin{example}
    John had 4 apples. \underline{John gave Jack some of his apples.} John has 1 apple now. John gave Jack 3 apples.
    \end{example}
    ~\\
    Find all the sentences that do not have numbers.\\
    ~\\
    Achieved by using \emph{Named Entity Recogniton}
\end{frame}

\begin{frame}{Preprocessing}
    \begin{block}{Other Steps}
    \begin{itemize}
    \item<1> Sentence Splitting\\
    not trival as it looks!
    \item<2-> Question Answering
    \begin{itemize}
        \item<2-> Replaced with manual answering
    \end{itemize}
    \end{itemize}
    \end{block}
    \only<1>{
    \begin{example}
    \underline{John spent 5}.\underline{20 dollars}.
    \end{example}}
    \only<2>{
    \begin{example}
    John has \alert{4} apples. Jack has \alert{5} apples. \underline{How many apples do they have in total?}
    \end{example}}
\end{frame}

\subsection{Question Generating}
\begin{frame}{Question Generating}
    \structure{Detect components of a sentence and raise a question about the numeric part.}\\ \pause
    Rule-based detection and template-based question generation adopted from \cite{Das2016}\\
    ~\\
    \begin{itemize}
    \item Clause detection rule
    \begin{itemize}
        \item find \textbf{subject phrases} and \textbf{verb phrases}
        \item find \textbf{auxiliary word} for the template
    \end{itemize}
    \item Numeric Phrase detection rule
    \begin{itemize}
        \item find \textbf{numerical phrases}
    \end{itemize}
    \item Question template of \emph{How-many}
    \end{itemize}
    ~\\
    Achieved by using \emph{Part-Of-Speech tags pattern matching}
\end{frame}

\begin{frame}{Question Generating}
    \begin{overlayarea}{\textwidth}{2cm}
    \begin{itemize}
    \only<1>{
    \item Clause detection rule (a \emph{POS tag pattern})\\
    $\{<DT>?<JJ.?>*<NN.?|PRP|PRP\$|POS|IN|DT|CC|VBG|VBN>+<RB.?|VB.?|MD|RP>+\}$}
    \only<2>{
    \item Numeric Phrase detection rule (a \emph{POS tag pattern})
    $\{<DT>?<CD>+<RB>?<JJ|JJR|JJS>?<NN|NNS|NNP|NNPS|VBG>+\}$}
    \only<3>{
    \item Question template of \emph{How-many}
    \begin{itemize}
        \item ”How many” + part of the numeric phrase excluding cardinal number + auxiliary word + subject part + remaining words of verb phrase + remaining part of the sentence as it is + ”?”
    \end{itemize}}
    \end{itemize}
    \end{overlayarea}
    \begin{example}
    \only<1>{\underline{John(\textbf{NNP}) has(\textbf{VBZ})} 4(CD) apples(NNS) .}
    \only<2>{John(NNP) has(VBZ) \underline{4(\textbf{CD}) apples(\textbf{NNS})} .}
    \only<3>{John(NNP) has(VBZ) 4(CD) apples(NNS) .}
    \begin{description}[Question]
        \item<1,3>[Clause] John has
        \begin{description}[]
        \item[Subject] John
        \item[Verb] has
        \item[Auxiliary word] does
        \end{description}
        \item<2,3>[Numeric] 4 apples
        \item<3>[Question] \underline{How many} \underline{apples} \underline{does} \underline{John} \underline{have}?
    \end{description}
    \end{example}
\end{frame}

\subsection{Postprocessing}
\begin{frame}{Postprocessing}
    \begin{itemize}
    \item Combine questions and sentences to form a new math word problem
    \item Reorder sentences if necessary (pronouns removal failed)
    \end{itemize}
    \begin{example}
    Input: He(\textbf{?}) gave Jack some of his apples. \underline{John had 4 apples.} He has 1 apple now. How many apples did he give to Jack?\\
    \structure{Move the underlined sentence to the first position}\\
    Output: \underline{John had 4 apples.} He(\textbf{John}) gave Jack some of his apples. He has 1 apple now. How many apples did he give to Jack?\\
    \end{example}
\end{frame}

\section{Experiment \& Analysis}

\subsection{Implementation}
\begin{frame}
    \frametitle{Implementation\footnote[frame,1]{source code available at: \href{https://github.com/xiaoyuin/ProjectMathWordProblemGeneration}{\structure{\emph{this link}}}}}
    \begin{itemize}
    \item Kotlin and Java programming language
    \item Stanford CoreNLP v3.8.0 
    \begin{itemize}
        \item Stanford Named Entity Recognizer
        \item Stanford Coreference Resolution
        \item Stanford POS Tagger
    \end{itemize}
    \item Intellij Idea IDE
    \item macOS High Sierra 10.13.3
    \end{itemize}
\end{frame}

\subsection{Experiment}
\begin{frame}{Experiment}
    \begin{block}{Data}
    20 math word problems were selected from MAWPS\footnote[frame, 1]{available at: \href{http://lang.ee.washington.edu/MAWPS/}{http://lang.ee.washington.edu/MAWPS/}}\\
    \textbf{Selection criteria}:
    \begin{itemize}
    \item 10 simple problems + 10 complex problems
    \item equation contains only addition and subtraction
    \item only problems with how-many questions
    \end{itemize}
    The question in each problem was manually transformed into answer form
    \end{block}
    \begin{block}{Simple problem}
    average number of words in each sentence $\leq$ 8
    \end{block}
\end{frame}

\begin{frame}{Experiment}
    \begin{example}[Original data]
    \begin{description}
        \item[simple] Keith has 20 books . Jason has 21 books . How many books do they have together ?  | ["41"]\\
        \textsl{(average 6 tokens per sentence)}
        \item[complex] Mary picked 122 oranges and Jason picked 105 oranges from the orange tree . How many oranges were picked in total ?  | ["227"]\\
        \textsl{(average 11 tokens per sentence)}
    \end{description}
    \end{example}
\end{frame}

\begin{frame}{Experiment}
    \begin{example}[After manual transformation]
    \begin{description}
        \item[simple] Keith has 20 books. Jason has 21 books. \textbf{They have 41 books together.}
        \item[complex] Mary picked 122 oranges and Jason picked 105 oranges from the orange tree. \textbf{227 oranges were picked in total.}
    \end{description}
    \end{example}
\end{frame}

\begin{frame}{Experiment}
    \begin{figure}
    \includegraphics[height=6cm]{"Modules overview".png}
    \end{figure}
\end{frame}

\subsection{Results and Analysis}

\begin{frame}[allowframebreaks]{Results and Analysis}
    \begin{center}
    \begin{tabular}{| l | l | l | l |}
    \hline
        & Failure & Success & Problems generated \\ \hline
        Simple problems & 0 & 10 & 22 \\ \hline
        Complex problems & 3 & 7 & 14 \\ \hline
    \end{tabular}
    \end{center}
    \begin{itemize}
    \item The system performs worse when facing with more complex problems
    \end{itemize}
    \pagebreak
    \begin{block}{Known limitation}
    \begin{itemize}
        \item Support only one question type
        \begin{itemize}
        \item Generate similiar questions (some of them with grammar errors)
        \item Unable to deal with multiple numbers.\\ e.g. John spent 5 dollars on 4 apples.
        \end{itemize}
        \item Performance was limited by how far the library can do\\
        i.e. If the library is not able to split the sentences, detect the tags, or resolve the coreference correctly, then the system would fail
    \end{itemize}
    \end{block}
    \pagebreak
    \begin{block}{Potential limitation}
    \begin{itemize}
        \item Can't identify \textbf{distraction}\\
        e.g. irrelevant numerical information
    \end{itemize}
    \end{block}
\end{frame}

\section{Future Work}

\begin{frame}{Future Work}
    \begin{itemize}
    \item Implement \textbf{Question Answering} in preprocessing to reduce the human efforts testing large data~\\
    ~\\
    \item Extend the \textbf{Rules and Templates} in question generation phase\\
    Enhance the ability to deal with variation of sentences\\
    ~\\
    \item Add \textbf{Understanding module} to avoid distractions
    \end{itemize}
\end{frame}

\section{Summary}

\begin{frame}{Summary}
    \begin{itemize}
    \item Related fields
    \begin{itemize}
        \item Math word problem generation
        \item Question generation
    \end{itemize}
    \item Three primary tasks
    \begin{itemize}
        \item Preprocessing
        \item Question Generating
        \item Postprocessing
    \end{itemize}
    \item Implementation and Experiment
    \begin{itemize}
        \item NLP techniques used
        \item 20 problems test
        \item Limitations
    \end{itemize}
    \item Future Work
    \begin{itemize}
        \item Question Answering
        \item More rules and templates
    \end{itemize}
    \end{itemize}
\end{frame}

\begin{frame}{Reference}
\bibliographystyle{amsalpha}
\bibliography{bibliography.bib}
\end{frame}

\end{document}
    